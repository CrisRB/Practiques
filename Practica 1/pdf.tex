\documentclass{article}

\usepackage[utf8]{inputenc}
\usepackage[T1]{fontenc}
\usepackage[catalan]{babel}
\usepackage{amsmath, amssymb, amsthm}
\usepackage{graphicx}
\usepackage[colorlinks,linkcolor=blue,citecolor=blue,urlcolor=blue]{hyperref}

\renewcommand{\baselinestretch}{1.5}

\title{Pràctica 1: Errors}
\author{Cristina Rosell Blanco: 1457235  \\ Oriol Ventosa Altimira: 1457285}
\date{11 de març de 2018}
\begin{document}
	\maketitle
	
	\newpage

	\section{Problema 1}
	\paragraph{A:} Volem veure que $0\leqslant f(x) \textless \frac{1}{2}$ per cada $x\neq0$. 
	
	Primer mirarem la desigualtat $0\leqslant f(x)$ :
	$$\frac{1-cos(x)}{x^2} \geqslant 0$$
	
	Veiem que la funció $cos(x)$ està fitada per 1, de manera que mai podrem obtenir un nombre negatiu al numerador. Com que $x^2$ és sempre positiu tampoc podrem tenir un nombre negatiu al denominador per tant la funcio $f(x)$ serà sempre positiva o igual a 0.
	
	Ara analitzem l'altra desigualtat:
	$$\frac{1-cos(x)}{x^2} \textless \frac{1}{2}$$
	$$-cos(x) \textless -1+\frac{x^2}{2}$$
	$$cos(x) \textgreater 1-\frac{x^2}{2}$$	
	
	Si desenvolupem $cos(x)$ amb la seva fórmula de Taylor a prop del punt 0, observem el següent:
	
	$$1-\frac{x^2}{2}+\frac{x^4}{4!}-\frac{x^6}{6!}+\dots \textgreater 1-\frac{x^2}{2}$$
	$$\frac{x^4}{4!}-\frac{x^6}{6!}+\dots \textgreater 0$$
	
	I aquesta última desigualtat serà sempre positiva per qualsevol $x\neq0$.
	
	\paragraph{B:} Si analitzem el punt $x_0=1,2\times 10^{-5}$ amb el programa de precisió simple obtenim que $x=0.000000$, en canvi, amb el programa de precisió doble obtenim que $x=0.4999997329749008$. 
	
	Com podem observar, el resultat obtingut amb el programa amb precisió simple és completament erroni. Això és degut a que la funció cosinus quan és treballa amb la variable $float$ aproxima $cos(1,2\times 10^{-5})$ a 1 fent que el numerador sigui $1-1$, donant així el resultat $f(x_0=1,2\times 10^{-5})=0$.
	
	\paragraph{D:} Quan x tendeix a 0 veiem que $cos(x)$ tendeix a 1. Llavors, al efectuar el calcul $1-cos(x)$ es produeix un error de cancel·lació ja que el mòdul de 1 i de $cos(x)$ prenen el mateix valor però amb signe oposat. En canvi, si utilitzem la nova fórmula $\frac{sin^2(x)}{x^2(1+cos(x))}$, obtinguda a partir de multiplicar per $\frac{1+cos(x)}{1+cos(x)}$, veiem que es comporta com $\frac{x^2}{x^2(1+cos(x))}$ quan x tendeix a 0 i ens queda finalment $\frac{1}{1+cos(x)}$. 
	
	\newpage
	
	\section{Problema 2}
	\paragraph{A:} Observem que quan $b^2 \gg 4ac$, $\sqrt{b^2-4ac}\sim|b|$. Llavors, si substituim aquesta fórmula a l'equació obtenim $\frac{-b\pm|b|}{2a}$. A partir d'aquí, si $b\textgreater0$, tindríem problemes per calcular la solució ja que es produiria un error de cancel·lació. I si $b\textless0$, tindríem el mateix problema però únicament amb la solució $\frac{-b-|b|}{2a}$.
	
	\paragraph{B:} A partir de la fórmula inicial: $\frac{-b\pm\sqrt{b^2-4ac}}{2a}$, analitzarem els casos positiu i negatiu per separat.
	
	Prenent + : $$\frac{-b+\sqrt{b^2-4ac}}{2a}$$ 
	veiem que en aquest cas quan $b<0$ no tenim cap problema, però quan $b>0$ tenim un error de cancel·lació. Així doncs, per solucionar-ho, ho multipliquem per $\frac{b+\sqrt{b^2-4ac}}{b+\sqrt{b^2-4ac}}$ i observem que 
	$$\frac{-b+\sqrt{b^2-4ac}}{2a}\cdot\frac{b+\sqrt{b^2-4ac}}{b+\sqrt{b^2-4ac}}=-\frac{2c}{(b+\sqrt{b^2-4ac})}$$
	Amb això evitem l'error de cancel·lació que teníem quan $b\textgreater0$.
	
	Prenent $-$ : $$-\frac{b+\sqrt{b^2-4ac}}{2a}$$
	en el cas de $b>0$ podem utilitzar la fórmula, però quan $b<0$ apareix un error de cancel·lació. Llavors, si ho multipliquem per $\frac{b-\sqrt{b^2-4ac}}{b-\sqrt{b^2-4ac}}$, ens queda 
	$$-\frac{b+\sqrt{b^2-4ac}}{2a}\cdot\frac{b-\sqrt{b^2-4ac}}{b-\sqrt{b^2-4ac}}=-\frac{2c}{(b-\sqrt{b^2-4ac})}$$
	En aquest cas evitem l'error de cancel·lació quan $b\textless0$, com volíem.
	
	\newpage
	
	\section{Problema 3}
	\paragraph{B:} Considerant el vector $x={\{10000,10001,10002\}}^T$, la variança en precisió simple ens dóna 1.000000 si ho calculem en dos bucles i 0.000000 si ho calculem en un sol bucle. En canvi, al calcular-la en precisió doble, tant si ho fem amb un bucle com amb dos, ens dóna 1 en els dos casos. Això es deu a que quan es calculen els quadrats de nombres tan grans en precisió simple, no dóna totes les xifres significatives correctes, és un error d'operació.
	Per exemple, quan calculem $(10002)^2$ obtenim com a resultat 100040000, quan realment hauríem d'obtenir 100040004.
	
	\paragraph{C:} Si calculem la variança d'un vector de 100 components, com el de l'arxiu \textit{prova100.txt}, veiem que la diferència entre els calculs amb un bucle o dos és bastant considerable, ja que amb dos bucles obtenim 0.696970 i amb un bucle ens surt 41.373737. I com hem esmentat abans, al calcular-ho en precisió doble, en ambdós casos obtenim el resultat de 0.696969696969697, molt més precís que amb precisió simple.
	
	\newpage
	
	\section{Problema 4}
	\paragraph{C:} Hem d'analitzar els dos programes per separat ja que es comporten de manera diferent.
    
    Pel que fa el programa que treballa amb variables de precisió simple, tant amb el càlcul de manera creixent com decreixent del sumatori, arribem a obtenir només quatre decimals significatius fent milers iteracions i obtenir 5 decimals significatius requereix 250000 iteracions.
    
    Pel que fa el programa que treballa amb variables de precisió doble obtenim les mateixes xifres significatives que amb el programa de precisió simple. Apart d'això els decimals no significatius varien quan comparem el cas creixent amb el decreixent.
    
    Això és degut a que, tot i que en paper haurien de donar el mateix resultat, el ordinador no és capaç d'efectuar certes propietats dels reals, com ara la propietat associativa o la commutativa, ja que treballa amb una aproximacions, limitades per l'epsilón de màquina.
    
    \paragraph{D:} El nou mètode que proposem per calcular $\frac{\pi^2}{6}$ és calcular $\pi$, implementant una aproximació de Beeler et al. al 1972 que correspon a la formula:
    $$2(1+\frac{1}{3}(1+\frac{2}{5}(1+...(1+\frac{n}{2n+1})...)))$$
    
    Amb aquest mètode obtenim totes les xifres representades a la figura 6 del pdf d'enunciats de manera que el mètode convergeix molt més ràpidament a $\frac{\pi^2}{6}$ que la formula usada ens els apartats anteriors ( $\sum_{k=1}^{n} \frac{1}{k^2}$ ).
    
    Cal destacar que amb el nou mètode quan fem un nombre relativament gran de iteracions, a partir de 57, obtenim resultats totalment erronis, ja que l'implementació de l'aproximació de $\pi$ que hem usat fa servir nombres enter de manera que arriba un punt que ja no es capaç de guardar tants dígits(per retrassar el problema s'han utilitzat el tipus de variable $long$ $long$ $ int$).
 

\end{document}
\end{document}
