\documentclass{article}

\usepackage[utf8]{inputenc}
\usepackage[T1]{fontenc}
\usepackage[catalan]{babel}
\usepackage{amsmath, amssymb, amsthm}
\usepackage{graphicx}
\usepackage[colorlinks,linkcolor=blue,citecolor=blue,urlcolor=blue]{hyperref}

\renewcommand{\baselinestretch}{1.5}

\title{Pràctica 2: Zeros de funcions}
\author{Cristina Rosell Blanco: 1457235  \\ Oriol Ventosa Altimira: 1457285}
\date{11 de març de 2018}
\begin{document}
	\maketitle
	
	\newpage

	\section{Problema 1}
	\paragraph{A:} Tant en precisió simple com en precisió doble, obtenim un error fitat per 0.002 i conseqüentment, l'aproximació de l'arrel és només de 2 xifres significatives.
	
	\paragraph{B:} Fent el mètode de Newton amb precisió simple veiem que amb 5 iteracions ja obtenim l'arrel amb 8 xifres significatives, ja que l'error absolut és menor que $\frac{1}{2}10^{-9}$. En canvi, amb precisió doble són necessàries 7 iteracions per obtenir l'arrel amb 15 xifres significatives, és a dir, amb un error absolut menor de $\frac{1}{2}10^{-16}$.
	
	\paragraph{C:} Per veure que l'arrel $\beta$ és entre 2 i 8, avaluem la funció $f(x)=x^3-x-400$ a 2 i a 8. Veiem que $f(2)<0$ i que $f(8)>0$. Així doncs, per Bolzano, com que f(x) és contínua, veiem que hi ha una arrel en (2,8). 
	
	Utilitzant la fórmula de Cardano per discriminants positius, obtenim com a arrel real 7.413133144378662 amb un error absolut de 0.0278. Per tant, té 1 xifra significativa.
	
	Aproximant l'arrel a 12 xifres, ja que amb més és produeixen problemes a nivell de precisió de variables.
	
	\subparagraph{c1:} Per el mètode de la bisecció necessitem 43 iteracions.
	\subparagraph{c2:} Per el mètode de la secant necessitem 6 iteracions.
	\subparagraph{c3:} Per el mètode de Newton necessitem 10 iteracions.
	
	COMPARAR ORDRE DE CONVERGENCIA NÚMERICA
	
	\newpage
	
	\section{Problema 2}
	\paragraph{A:} Calculant 10 iteracions de la successió amb el valor inicial de 5.7, mirem els quocients $frac{e_{k}}{e_{k-1}^i}$ per cada iteració i obtenim els següents resultats:
	    \begin{figure}[h!]
		\begin{center}	
			\begin{tabular}{|c|c|c|c|}
				\hline $Ordres$ & 1 & 2 & 3 \\
				\hline 1 & \textbf{500(escó 1)} & \textbf{250(escó 3)} & \textbf{166(escó 5)} \\
				\hline B=300 & \textbf{300(escó 2)} & 150 & 100 \\
				\hline C=200 & \textbf{200(escó 4)} & 100 & 66 \\
				\hline D=100 & 100 & 50 & 33 \\
				\hline
			\end{tabular}
		\end{center}
	\end{figure} 
	
	I observem que l'únic valor que s'estabilitza és el de ordre 2.
	
	A PARTIR DE 5,7 FUNCIONA, ABANS NO. I ITERACIONS AMB 10 JA ÉS 0, MÉS ENDAVANT ÉS -NAN.
	
	\newpage
	
	\section{Problema 3}
	\paragraph{B:} 
	
	\newpage
	
	\section{Problema 4}

	
\end{document}