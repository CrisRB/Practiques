\documentclass{article}

\usepackage[utf8]{inputenc}
\usepackage[T1]{fontenc}
\usepackage[catalan]{babel}
\usepackage{amsmath, amssymb, amsthm}
\usepackage{graphicx}
\usepackage[colorlinks,linkcolor=blue,citecolor=blue,urlcolor=blue]{hyperref}

\renewcommand{\baselinestretch}{1.5}

\title{Pràctica 2: Zeros de funcions}
\author{Cristina Rosell Blanco: 1457235  \\ Oriol Ventosa Altimira: 1457285}
\date{11 de març de 2018}
\begin{document}
	\maketitle
	
	\newpage

	\section{Problema 1}
	\paragraph{A:} Tant en precisió simple com en precisió doble, obtenim un error fitat per 0.002 i conseqüentment, l'aproximació de l'arrel és només de 2 xifres significatives.
	
	\paragraph{B:} Fent el mètode de Newton amb precisió simple veiem que amb 5 iteracions ja obtenim l'arrel amb 8 xifres significatives, ja que l'error absolut és menor que $\frac{1}{2}10^{-9}$. En canvi, amb precisió doble són necessàries 7 iteracions per obtenir l'arrel amb 15 xifres significatives, és a dir, amb un error absolut menor de $\frac{1}{2}10^{-16}$.
	
	\paragraph{C:} Per veure que l'arrel $\beta$ és entre 2 i 8, avaluem la funció $f(x)=x^3-x-400$ a 2 i a 8. Veiem que $f(2)<0$ i que $f(8)>0$. Així doncs, per Bolzano, com que f(x) és contínua, veiem que hi ha una arrel en (2,8). 
	
	Utilitzant la fórmula de Cardano per discriminants positius, obtenim com a arrel real 7.413133144378662 amb un error absolut de 0.0278. Per tant, té 1 xifra significativa.
	
	Aproximant l'arrel a 12 xifres, ja que amb més és produeixen problemes a nivell de precisió de variables.
	
	\subparagraph{c1:} Per el mètode de la bisecció necessitem 43 iteracions.
	\subparagraph{c2:} Per el mètode de la secant necessitem 6 iteracions.
	\subparagraph{c3:} Per el mètode de Newton necessitem 10 iteracions.
	
	El mètode de bisecció és el que té menor ordre de convergència, ja que aproxima un decimal cada dos o tres iteracions.
	
	El mètode de Newton, convergeix en 10 iteracions ja que, tot i donar de 4 a 5 decimals quan és prop de l'arrel, al començar per $x_0=2$ s'allunya molt de l'arrel a les primeres iteracions.
	
	El mètode de la secant convergeix de 6 iteracions sent així el més ràpid en aquestes condicions.
	
	\newpage
	
	\section{Problema 2}
	\paragraph{A:} Calculant 10 iteracions de la successió amb el valor inicial de 5.7, mirem els quocients $\frac{e_{k}}{(e_{k-1})^i}$ per cada iteració i obtenim els següents resultats:
	    \begin{figure}[h!]
		\begin{center}	
			\begin{tabular}{|c|c|c|c|}
				\hline Iteració & Ordre 1 & Ordre 2 &Ordre 3 \\
\hline 1 & 0.01248918081925201 & \textbf{0.005463913951181786 }& 0.002390417441942918 \\
\hline 2 & 1.842692755025232 & \textbf{64.54890401705153 }& 2261.126277530449 \\
\hline 3 & 1.702255038608466 & \textbf{32.35993528700995 }& 615.1636435369242 \\
\hline 4 & 1.46280082950094 & \textbf{16.33591953962786 }& 182.4324007911864 \\
\hline 5 & 1.104668372244713 & \textbf{8.433446801228495 }& 64.3840511199098 \\
\hline 6 & 0.6698905392401661 & \textbf{4.629617534898694 }& 31.99531455355764 \\
\hline 7 & 0.2824962810940362 & \textbf{2.914406725843755 }& 30.06682612156564 \\
\hline 8 & 0.06405272718457172 & \textbf{2.339172858148314 }& 85.42539718146021 \\
\hline 9 & 0.00431413316935736 & \textbf{2.459691339153118 }& 1402.3863535038 \\
\hline 10 & 2.15632745224664E-05 & \textbf{2.849759631796303 }& 376618.5859459582 \\ \hline
			\end{tabular}
		\end{center}
	\end{figure} 
	
	I observem que l'únic valor que s'estabilitza és el d'ordre 2.
	
	\newpage
	
	\section{Problema 3}
	\paragraph{B:} Per veure que l'ordre de convergència de la successió és 3 utilitzarem el mateix mètode que el problema anterior, en aquest cas observant com evolucionen els quocients dels ordres 2, 3 i 4.
	
	\begin{figure}[h!]
		\begin{center}	
			\begin{tabular}{|c|c|c|c|}
				\hline Iteració & Ordre 2 & Ordre 3 &Ordre 4 \\
\hline 1 & 0.001428556504355381 & \textbf{4.081604093779827E-06 }& 1.166176621475511E-08 \\
\hline 2 & 0.002856913665404192 & \textbf{1.632550413993405E-05 }& 9.329021336922217E-08 \\
\hline 3 & 0.005710795230583281 & \textbf{6.527373165221887E-05 }& 7.460712338254702E-07 \\
\hline 4 & 0.01137455278196829 & \textbf{0.0002602084821147401 }& 5.952625607557341E-06 \\
\hline 5 & 0.02203434390101422 & \textbf{0.001013771390866927 }& 4.664229793077517E-05 \\
\hline 6 & 0.0353760743095244 & \textbf{0.003398516082247287 }& 0.0003264893515384728 \\
\hline 7 & 0.02787719417111068 & \textbf{0.007272752467876818 }& 0.001897354810328137 \\
\hline 8 & 0.004662467037722478 & \textbf{0.01138324824668134 }& 0.0277917976893345 \\
\hline 9 & 9.631885813428722E-06 & \textbf{0.01231392250244575 }& 15.74278291223454 \\
\hline 10 & 0 & \textbf{0 }& 0 \\

				\hline
			\end{tabular}
		\end{center}
	\end{figure}
	
	Observem així que l'únic ordre que no tendeix a 0 o infinit és l'ordre 3.
	 
	\newpage
	
	\section{Problema 4}
	
	Abans d'analitzar l'ordre de convergència hem mirat a partir de quina iteració l'error absolut és major a l'error absolut de l'iteració anterior, en aquest cas a partir de la iteració 5 el error comença a degenerar.
	
	També observem que la velocitat de convergència a l'iteració quatre disminueix molt respecte la tercera iteració i per tant el càlcul de l'ordre de convergència allà tampoc té sentit.
	
	De manera que el càlcul de l'ordre de convergència que hem utilitzat en els exercicis anteriors només el podem aplicar en les tres primeres iteracions.
	
	Per verificar que és d'ordre quadràtic fem la taula d'ordres de 1, 2 i 3 (amb tres iteracions):
	
	\begin{figure}[h!]
		\begin{center}	
			\begin{tabular}{|c|c|c|c|}
				\hline Iteració & Ordre 1 & Ordre 2 &Ordre 3 \\
				\hline 1 & INF & \textbf{4.931909793448288E+17 }& 1.613490835259473E+27 \\
				\hline 2 & 0.001901903831424686 & \textbf{0.04127396441818221 }& 0.8957025642654479 \\
				\hline 3 & 3.487768896996442E-06 & \textbf{0.03979667415618414 }& 454.0940987395674 \\
				\hline
			\end{tabular}
		\end{center}
	\end{figure}
	
	Observem que l'únic ordre estable és dos, i per tant la convergència és quadràtica.
	
	El mal condicionament del problema és degut al càlcul de la successió $c_k$.
	
	Es produeix un error de cancel·lació al restar els quadrats de $a_k$ i $b_k$, que són definides com a la mitjana aritmètica, $a_k$, i geomètrica, $b_k$, dels valors de la iteració anterior. 
	
	Com que és fan diferents mitjanes dels mateixos valors, en poques iteracions els valors de $a_k$ i $b_k$ acaben sent molt similars, produint l'error de cancel·lació que comentavem.
	
	
	
\end{document}
